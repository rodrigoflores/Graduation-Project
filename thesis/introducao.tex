
O ponto de partida para este projeto é a necessidade de se fazer processamentos custosos de 
dados em bioinformática. Estes processamentos costumam ser combinatórios, o que os fazem
demorar um bom tempo para serem executados em um computador comum, sendo necessário utilizar recursos
computacionais de alta performance para obter os resultados em tempo hábil.

\subsection{Programação paralela e distribuída}

Uma abordagem clássica para se resolver esse problema é utilizar \textit{clusters}, que são um grupo de computadores
ligados entre si de modo a parecer ser um único computador muito mais potente. \textit{Clusters} podem ser tanto máquinas
específicas para isso, produzidas com um alto custo e com um hardware específico para otimizar seu desempenho, ou 
pode ser utilizado o processamento em grade: utilizando computadores pessoais que são produzidos em massa a um preço mais baixo e 
trabalhando em paralelo para fazer o processamento. 

O projeto \emph{Beowulf} é um dos exemplos de computação em grade: computadores pessoais baratos constituem uma grade dedicada 
que funciona como um super computador. De acordo com o projeto Beowulf, uma rede deste tipo provê o mesmo recurso computacional 
que um super computador mas custando de um décimo a um terço do preço. Outro exemplo de computação em grade são os 
projetos de computação oportunista ou voluntária, 
não necessariamente usam computadores de forma dedicada mas utilizam o tempo ocioso do computador.

\subsection{Computação oportunista}

A computação oportunista é um tipo de computação distribuída no qual pessoas que possuem computadores podem doar processamento e
armazenamento ocioso de suas máquinas. Estes projetos normalmente têm um objetivo bem definido com um apelo humanitário.

O primeiro projeto de Computação voluntária foi o \textit{Great Internet Mersenne Prime Search}, lançado em janeiro de 1996. 
Seu objetivo era encontrar números primos de Mersenne\footnote{Isto é, números primos na forma $M_n = 2^n - 1$}. Em seguida houveram 
muitos outros projetos. Entre os projetos mais destacados podemos citar 
o Folding At Home, que investiga o enrolamento de proteínas e que pode ajudar o desenvolvimentos de pesquisas contra 
Câncer, doença de Huntington, entre outras. Outro dos projeto notável de computação voluntária, e de grande importância 
para este trabalho, é o \textit{SETI@Home} que atraiu centenas de milhares de voluntários 
de vida inteligente fora da Terra. Como um produto deste projeto, nasceu o \emph{middleware} 
\emph{BOINC} que hoje é utilizado em diversos projetos. 

\subsection{Linguagens Interpretadas}


Uma das possíveis divisões para linguagens de programação é se seus códigos são compilados para código de máquina ou se são simplismente 
interpretados. Enquanto no primeiro caso há a figura de um compilador, que transforma o código fonte em código de máquina 
para ele poder então ser executado, no segundo caso,
há a figura de um interpretador que não converte o programa para 
código de máquina, mas sim o interpreta e o executa. 

Linguagens interpretadas possuem vantagens e desvantagens sobre as compiladas: embora elas sejam mais fáceis 
de serem multiplataforma (basta o interpretador
estar disponível para aquela plataforma) e permitam escopo e tipagem dinâmica, também costumam ser menos eficientes que linguagens compiladas e 
a presença de um interpretador é obrigatória para sua execução. 

Dentre as linguagens interpretadas, uma que adquiriu destaque na área de estatística e bioinformática é a linguagem \emph{R}. Esta 
linguagem já vem com as distribuições estatísticas mais usuais, possui muitos dos algoritmos normalmente utilizados já implementados e
 uma extensibilidade poderosa, com bibliotecas que podem ser baixadas facilmente.


\subsection{Linguagens interpretadas e computação distribuída}

Embora já existam ótimas soluções para a execução distribuída de programas compilados como o
 MPI\footnote{disponível em \url{http://www.mcs.anl.gov/research/projects/mpi/}}, não se fala muito em soluções 
para execução distribuída de programas em linguagem interpretada. 
Para a linguagem \emph{R} há um pacote chamado \emph{gridR} que permite o uso do R com o \emph{Condor}, %Colocar note
um middleware para execução de programas em grades.  
Um outro trabalho que relaciona o R com computação distribuída é o \cite{Dias} que 
utiliza o \emph{Alchemi}, um \textit{middleware} para processamento em grade e baseado na 
tecnologia \textit{.NET}, e a interface \textit{COM} junto com o pacote do \emph{R}
\emph{RCom}.  


O artigo \cite{boinc} fala sobre a utilização do Middleware de computação voluntária \emph{BOINC} como solução para computação 
em grade na Universidade de Extremadura na Espanha e dentre os programas executados na grade, haviam
programas em R. Porém isso foi somente instalado em redes de computadores cujo sistema
operacional é o Linux. 

Seguindo este feito, este trabalho tem a intenção de construir algo semelhante nos laboratórios da rede CEC do IME/USP. Utilizando
não somente os computadores executando Linux, mas também os computadores cujo sistema operacional é o Microsoft Windows %Colocar marca registrada.
já que estes constituem uma parte considerável da rede.  



