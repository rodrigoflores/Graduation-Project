
O ponto de partida para este projeto é a necessidade de se fazer processamentos custosos de 
dados em bioinformática. Estes processamentos costumam ser combinatórios, o que os fazem
demorar um bom tempo para serem executados em um computador comum, sendo necessário utilizar recursos
computacionais de alta performance para obter os resultados em tempo hábil.

\subsection{Programação paralela e distribuída}

Uma abordagem clássica para se resolver esse problema é utilizar \textit{clusters}, que são um grupo de computadores
ligados entre si de modo a parecer ser um único computador muito mais potente. \textit{Clusters} podem ser tanto máquinas
específicas para isso, produzidas com um alto custo e com um hardware específico para otimizar seu desempenho, ou 
pode ser utilizado o processamento em grade: utilizando computadores pessoais que são produzidos em massa a um preço mais baixo e 
trabalhando em paralelo para fazer o processamento. 

O projeto \emph{Beowulf} é um dos exemplos de computação em grade: computadores pessoais baratos constituem uma grade dedicada 
que funciona como um super computador. De acordo com o projeto Beowulf, uma rede deste tipo provê o mesmo recurso computacional 
que um super computador mas custando de um décimo a um terço do preço. Outro exemplo de computação em grade são os 
projetos de computação oportunista ou voluntária, 
não necessariamente usam computadores de forma dedicada mas utilizam o tempo ocioso do computador.

\subsection{Computação oportunista}

A computação oportunista é um tipo de computação distribuída no qual pessoas que possuem computadores podem doar processamento e
armazenamento ocioso de suas máquinas. Estes projetos normalmente têm um objetivo bem definido com um apelo humanitário.

O primeiro projeto de Computação voluntária foi o \textit{Great Internet Mersenne Prime Search}, lançado em janeiro de 1996. 
Seu objetivo era encontrar números primos de Mersenne\footnote{Isto é, números primos na forma $M_n = 2^n - 1$}. Em seguida houveram 
muitos outros projetos. Entre os projetos mais destacados podemos citar 
o Folding At Home, que investiga o enrolamento de proteínas e que pode ajudar o desenvolvimentos de pesquisas contra 
Câncer, doença de Huntington, entre outras. Outro dos projeto notável de computação voluntária, e de grande importância 
para este trabalho, é o \textit{SETI@Home} que atraiu centenas de milhares de voluntários 
de vida inteligente fora da Terra. Como um produto deste projeto, nasceu o \emph{middleware} 
\emph{BOINC} que hoje é utilizado em diversos projetos. 

\subsection{Linguagens Interpretadas}

Uma das possíveis divisões para linguagens de programação é se seus códigos são compilados para código de máquina ou se são simplesmente 
interpretados. Enquanto no primeiro caso há a figura de um compilador, que transforma o código fonte em código de máquina 
para ele poder então ser executado, no segundo caso,
há a figura de um interpretador que não converte o programa para 
código de máquina, mas sim o interpreta e o executa. 

Linguagens interpretadas possuem vantagens e desvantagens sobre as compiladas: embora elas sejam mais fáceis 
de serem multiplataforma (basta o interpretador
estar disponível para aquela plataforma) e permitam escopo e tipagem dinâmica, também costumam ser menos eficientes que linguagens compiladas e 
a presença de um interpretador é obrigatória para sua execução. 

Dentre as linguagens interpretadas, uma que adquiriu destaque na área de estatística e bioinformática é a linguagem \emph{R}, que 
será de fundamental importância para este trabalho. 


\subsubsection{Linguagem R}

A linguagem \emph{R} é uma linguagem interpretada bastante utilizada no desenvolvimento de rotinas de bioinformática e na análise de
dados. Muitas funções normalmente utilizadas em análises de dados vêm incluídas na linguagem como por exemplo cálculo da
 média, desvio padrão, ajuste de curva e etc. A geração de gráficos do \emph{R} também é bastante elaborada e é possível 
gerar gráficos dos mais diversos tipos. Também há implementações de algoritmos mais elaborados de estatística como por 
exemplo algoritmos de \emph{clustering} que também são bastante usados em bioinformática.

Outro ponto interessante da linguagem \emph{R} são os pacotes extras que se pode baixar. Ao todo, no repositório oficial
são $2076$ pacotes, para os mais diversos propósitos que podem ser desde funções mais específicas para análise como 
por exemplo o \emph{Bayesclust}\footnote{Disponível em \url{http://cran.r-project.org/web/packages/bayesclust/index.html}}, 
para geração de gráficos utilizando outras bibliotecas como por exemplo o 
\emph{rgl}\footnote{Disponível em \url{http://cran.r-project.org/web/packages/rggobi/index.html}}. 

Com maior destaque para área de bioinformática, o artigo 
\cite{bioconductor} fala sobre o \emph{Bioconductor}, um projeto  que propõe o fornecimento 
de ferramentas para a análise de dados desta área. O projeto foi iniciado em 2001 e na grande maioria, 
seus pacotes são para o ambiente de programação da linguagem \emph{R}.

\subsection{Linguagens interpretadas e computação distribuída}

Embora já existam ótimas soluções para a execução distribuída de programas compilados como o
 MPI\footnote{disponível em \url{http://www.mcs.anl.gov/research/projects/mpi/}}, não se fala muito em soluções 
para execução distribuída de programas em linguagem interpretada. 
Para a linguagem \emph{R} há um pacote chamado \emph{gridR} que permite o uso do R com o \emph{Condor}, %Colocar note
um middleware para execução de programas em grades.  
Um outro trabalho que relaciona o R com computação distribuída é o \cite{Dias} que 
utiliza o \emph{Alchemi}, um \textit{middleware} para processamento em grade e baseado na 
tecnologia \textit{.NET}, e a interface \textit{COM} junto com o pacote do \emph{R}
\emph{RCom}.  


O artigo \cite{boinc} fala sobre a utilização do Middleware de computação voluntária \emph{BOINC} como solução para computação 
em grade na Universidade de Extremadura na Espanha e dentre os programas executados na grade, haviam
programas em R. Porém isso foi somente instalado em redes de computadores cujo sistema
operacional é o Linux. 


\subsection{Solução multiplataforma}

Embora o ambiente \emph{Linux} seja muito utilizado no ambiente acadêmico e em ambientes de desenvolvimento de softwares, é
muito difícil encontrá-lo em um ambiente doméstico ou de trabalho cotidiano. Embora a cada ano o número de usuários deste 
sistema cresça a cada ano, percebemos que o sistema normalmente utilizado mesmo é o Microsoft Windows. O artigo \cite{Dias}
dispõe uma solução usando o \emph{middleware} \emph{Alchemi}, baseado na tecnologia da Microsoft \emph{.NET} que é executado
no sistema operacional \emph{Microsoft Windows}. O artigo \cite{boinc} porém utiliza uma rede com computadores rodando \emph{Linux}

Porém como temos muitos computadores com sistemas \emph{Linux} e \emph{Windows} na rede \emph{CEC} do IME-USP,
pensamos que o ideal seria termos uma solução que pudesse utilizar os dois sistemas. Como tanto o \emph{BOINC} como o 
\emph{R} estão disponíveis para os dois sistemas, um dos objetivos deste trabalho é utilizar os dois sistemas na grade.

\subsection{Utilização do \emph{BOINC} com o  \emph{R}}

Embora a \emph{API} do \emph{BOINC} não esteja disponível para ser utilizada em programas escritos na linguagem \emph{R}, 
a utilização destas duas tecnologias juntas mostra algumas vantagens. Como ambas as tecnologias são de código aberto, não
se gasta na compra de licenças. A possibilidade de se executar rotinas inalteradas e de instalar pacotes
do \emph{R} sob demanda utilizando o \emph{wrapper} também foi um fator decisivo na escolha. A 
utilização bem sucedida do processamento de rotinas em \emph{R} na grade descrita no artigo \cite{boinc} também foi um fator 
determinante nesta decisão, além da disponibilidade multiplataforma de ambas as tecnologias.



 

