O ponto de partida para este projeto vem da necessidade de se fazer processamentos custosos de 
dados em bioinformática. Estes processamentos costumam ser combinatórios, o que os fazem
demorar um bom tempo sendo executados em um computador comum, sendo necessário utilizar recursos
computacionais de alta performance para obter os resultados em tempo hábil.

\subsection{Programação paralela e distribuída}

Uma abordagem clássica para se resolver esse problema é utilizar \textit{clusters} que são um grupo de computadores
ligados entre si de modo a parecer ser um único computador muito mais potente. \textit{Clusters} podem ser tanto máquinas
específicas para isso, produzidas com um alto custo e com um hardware específico para otimizar seu desempenho, ou 
pode ser utilizado o conceito de computação em grades: utilizar computadores comuns trabalhando em paralelo 
para fazer o processamento. 

Uma outra alternativa é utilizar grades computacionais distribuídas. Grades computacionais costumam ser constituídas
de um conjunto de computadores cujo principal objetivo não é fazer processamento pesado, mas sim lidar com o usuário 
cotidiano que normalmente utiliza a máquina para acessar a internet, ler seu correio eletrônico, elaborar
trabalhos usando processadores de texto e planilhas eletrônicas, entre outras coisas. Embora grades computacionais tenham um 
processamento menor que clusters dedicados, costumam ter um custo muito menor, já que normalmente computadores ``comuns'' já existem em
uma empresa ou universidade. 


\subsection{Computação voluntária}
%Bibliografia Wikipedia


A computação voluntária é um tipo de computação distribuída no qual pessoas que possuem computadores podem doar processamento e
armazenamento ocioso de suas máquinas enquanto elas estiverem ociosas. Estes projetos normalmente têm um objetivo bem definide.

O primeiro projeto de Computação voluntária foi o \textit{Great Internet Mersenne Prime Search}, lançado em janeiro de 1996. 
Seu objetivo era encontrar números primos de Mersenne\footnote{Isto é, números primos na forma $M_n = 2^n - 1$}. Em seguida houveram 
muitos outros projetos e alguns utilizam um apelo social, de modo a obter mais doadores de processamento, um exemplo disso é o
o Folding At Home, que investiga o enrolamento de proteínas e que pode ajudar o desenvolvimentos de pesquisas contra 
Câncer, doença de Huntington, entre outras. 

Um dos projetos mais notáveis de computação voluntária foi o \textit{SETI@Home}. Os objetivos iniciais deste projeto eram: 

\begin{itemize}
  \item Provar a viabilidade e a praticidade de grades computacionais distribuídas;
  \item Fazer um trabalho útil e apoiar uma análise de observações para detectar vida inteligente fora da Terra.
\end{itemize}

Este projeto atraiu centenas de milhares de voluntários, porém só o primeiro objetivo teve sucesso: não foram encontrados sinais 
de vida inteligente fora da Terra. Um middleware então foi criado para este fim: o \textit{BOINC}. 

\subsection{Linguagens Interpretadas}

%Colocar bibliografia da wikipedia

Uma das possíveis divisões para linguagens de programação é se seus códigos são compilados ou simplismente interpretados. Enquanto no primeiro caso
é necessária a utilização de um programa para transformar o código fonte em código de máquina para ele poder então ser executado. No segundo caso,
há a figura de um interpretador que não converte o programa para código de máquina, mas sim o interpreta e o executa. 

Linguagens interpretadas possuem vantagens e desvantagens: embora elas sejam mais fáceis de serem multiplataforma (basta o interpretador
estar disponível para aquela plataforma)e permitam escopo e tipagem dinâmica, também costumam ser menos eficientes que linguagens compiladas e 
a presença de um interpretador é obrigatória para sua execução. 

Dentre as linguagens interpretadas, uma que adquiriu destaque na área de estatística e bioinformática é a linguagem \emph{R}, que também 
pode ser utilizado como um ambiente. Esta linguagem já vem com as distribuições estatísticas mais usuais e possui uma extensibilidade 
forte com bibliotecas que podem ser baixadas facilmente.


\subsection{Linguagens interpretadas e computação distribuída}

Embora já existam ótimas soluções para a execução distribuída de programas compilados como o
 MPI\footnote{disponível em \url{http://www.mcs.anl.gov/research/projects/mpi/}}, não se fala muito em soluções 
para execução distribuída de programas em linguagem interpretada. 
Para a linguagem \emph{R} há um pacote chamado \emph{gridR} que permite o uso do R com o \emph{Condor}, %Colocar note
um middleware para execução de programas em grades.  Um outro trabalho que relaciona o R com computação distribuída é o %Colocar o artigo do Rodrigo


O artigo %Colocar bibliografia
sobre a utilização do Middleware de computação voluntária BOINC como solução para computação 
em grade na Universidade de Extremadura na Espanha. Dentre os programas executados na grade, haviam
programas em R. Porém isso foi somente instalado em redes de computadores com computadores cujo sistema
operacional é o Linux. 

Seguindo este feito, este trabalho tem a intenção de construir algo semelhante nos laboratórios da rede CEC do IME/USP. Utilizando
não somente os computadores executando Linux, mas também os computadores cujo sistema operacional é o Microsoft Windows %Colocar marca registrada.
já que estes são uma parte considerável da rede.  

