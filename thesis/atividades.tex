
O início do projeto deu-se ainda em 2008, com a visita ao colégio Rainha da Paz na Lapa, onde o aluno de mestrado do \textit{IME}
Rodrigo Assirati Dias mantém uma grade de computadores com o middleware \textit{Alchemi} citada no trabalho \cite{Dias}
. Nesta visita foi possível esclarecer dúvidas, entender o funcionamento da grade e receber algumas dicas quanto à manutenção da grade. 
Após esse encontro, começou-se a buscar alternativas para o a computação de alta performance com o \textit{R}. Um primeiro pacote encontrado
que fazia esta função foi o \textit{GridR}, que permite submeter rotinas do \textit{R} para \textit{clusters}, máquinas remotas e 
grades. Um dos arcabouços possíveis para o uso deste pacote é o Condor\footnote{Disponível em \url{http://www.cs.wisc.edu/condor/}}
, desenvolvido pela \textit{University of Wisconsin-Madison} e é bastante utilizado em empresas
de grande porte como a \textit{NASA} e pode ser executado tanto em sistemas
baseados em \textit{UNIX}, como em sistemas \textit{Windows}. Seguindo esta busca, encontramos o 
\textit{middleware} \textit{BOINC}, no artigo \cite{boinc}
sendo utilizado para um propósito semelhante em um trabalho na Universidade de Extremadura, na Espanha 
e decidimos que a abordagem seria interessante para nosso trabalho.

Escolhido o \textit{middleware} nos focamos na instalação do servidor. A própria página do \textit{BOINC} 
possui um guia de instalação do servidor do \textit{middleware} no sistema \textit{Debian GNU \ Linux} e por
esta distribuição \textit{Linux} ser bastante conhecida por sua estabilidade, foi instalado este sistema no servidor.
Instalado o servidor, o foco foi em ter uma aplicação em R executando remotamente em uma grade de computadores. 
Este processo no sistema Linux foi relativamente simples: utilizando o ``truque do \textit{shebang}'' é possível 
colocar um script como executável e o \textit{wrapper} executá-lo como se fosse um arquivo compilado. Já para
o sistema Windows %Colocar marca registrada
o trabalho foi mais complicado: havia um bug nas configurações de compilação do \textit{wrapper} e até perceber isso
atrasou bastante o andamento do projeto. Passado isso, foi necessário utilizar um programa escrito em C, que apenas executava 
o interpretador junto com o arquivo com a rotina em \textit{R}.

Finalizado esta parte, nos focamos na aplicação a ser executada na grade. Para isso foi criado um programa em \textit{R} que
apenas acessava um arquivo e fazia alguns cálculos custosos. Isto foi então configurado para o mesmo programa poder
ser executado tanto em sistemas Windows como em sistemas Linux. Paralelamente a isso, foi pedido para a administração
da rede do \textit{CEC} para instalar o \textit{BOINC} nas máquinas da rede. Como a rede estava em reforma, para a troca
do sistema operacional não foi possível concluir a instalação até o fim deste trabalho, mas acredito que em breve teremos a grade em 
pleno funcionamento.


