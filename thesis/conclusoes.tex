
A principal conclusão deste trabalho é que é possível a utilização do \emph{BOINC} para o processamento
de rotinas de bioinformática escritas na linguagem \emph{R}. A utilização multiplataforma
também foi de grande utilidade já que podemos incluir praticamente qualquer tipo de computador 
existente em redes. O custo total de implementação foi somente a inutilização de uma máquina, utilizada
como servidor da rede. Embora o \emph{BOINC} seja um projeto consolidado e possua uma documentação com bastante conteúdo,
as mensagens de erro, principalmente do \emph{Wrapper}, são muito pouco informativas, o que tornou o desenvolvimento 
deste projeto bastante trabalhoso.

Para a continuação do projeto há várias sugestões:

\begin{itemize}
  \item \emph{Benchmark} da rede e comparação com grade descrita no artigo \cite{Dias}
Para determinar a viabilidade, seria interessante estabelecermos a comparação com outra 
alternativa. 
  \item Comparação com redes ``alugadas'': hoje já existe oportunidade de se fazer esse tipo de processamento.
em grades alugadas como a oferecida pela empresa \emph{amazon}. Como os computadores na rede 
consomem energia elétrica seria interessante comparar o gasto da energia elétrica com o gasto em 
uma grade ``alugada''.
  \item Analisar o desempenho nas máquinas com Windows e com Linux. Seria interessante analisar 
o benchmark da grade em ambos os sistemas e determinar qual das duas plataformas é mais
propícia para o processamento;
  \item Utilização de máquinas virtuais: como feito no artigo \cite{boinc}, podemos utilizar
máquinas virtuais, que são iniciadas em cada nó e é feito o processamento. 
\end{itemize}


