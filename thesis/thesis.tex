\documentclass[a4paper,12pt,titlepage]{article}
\usepackage[T1]{fontenc}
\usepackage[brazil]{babel}
\usepackage[utf8x]{inputenc}
\usepackage{graphicx}
\usepackage{times}
\usepackage{ucs}
\usepackage{url}




\title{Trabalho de Conclusão de Curso \\ }


\author{Rodrigo L. M. Flores \and
        Orientador: Roberto Hirata Jr. }


\date{\today}

\begin{document}

\maketitle

\part{}

\section{Introdução}


O ponto de partida para este projeto é a necessidade de se fazer processamentos custosos de 
dados em bioinformática. Estes processamentos costumam ser combinatórios, o que os fazem
demorar um bom tempo para serem executados em um computador comum, sendo necessário utilizar recursos
computacionais de alta performance para obter os resultados em tempo hábil.

\subsection{Programação paralela e distribuída}

Uma abordagem clássica para se resolver esse problema é utilizar \textit{clusters}, que são um grupo de computadores
ligados entre si de modo a parecer ser um único computador muito mais potente. \textit{Clusters} podem ser tanto máquinas
específicas para isso, produzidas com um alto custo e com um hardware específico para otimizar seu desempenho, ou 
pode ser utilizado o processamento em grade: utilizando computadores pessoais que são produzidos em massa a um preço mais baixo e 
trabalhando em paralelo para fazer o processamento. 

O projeto \emph{Beowulf} é um dos exemplos de computação em grade mais famosos: computadores pessoais baratos constituem uma grade dedicada 
que funciona como um super computador. De acordo com o projeto Beowulf, uma rede deste tipo provê o mesmo recurso computacional 
que um super computador mas custando de um décimo a um terço do preço. Outro exemplo de computação em grade são os 
projetos de computação oportunista ou voluntária, 
não necessariamente usam computadores de forma dedicada mas utilizam o tempo ocioso do computador.

\subsection{Computação oportunista}

A computação oportunista é um tipo de computação distribuída no qual pessoas que possuem computadores podem doar processamento e
armazenamento ocioso de suas máquinas. Estes projetos normalmente têm um objetivo bem definido com um apelo humanitário (ou científico).

O primeiro projeto de Computação voluntária foi o \textit{Great Internet Mersenne Prime Search}, lançado em janeiro de 1996. 
Seu objetivo era encontrar números primos de Mersenne\footnote{Isto é, números primos na forma $M_n = 2^n - 1$}. Após este projeto, nasceram vários outros 
. Entre os projetos mais destacados podemos citar 
o \textit{Folding At Home}, que investiga o enrolamento de proteínas e que pode ajudar o desenvolvimentos de pesquisas contra 
Câncer, doença de Huntington, entre outras. Outro projeto notável de computação voluntária, e de grande importância 
para este trabalho, é o \textit{SETI@Home} que atraiu centenas de milhares de voluntários 
de vida inteligente fora da Terra. Como um produto deste projeto, nasceu o \emph{middleware} 
\emph{BOINC} que hoje é utilizado em diversos projetos. 

\subsection{Linguagens Interpretadas}

Uma das possíveis divisões para linguagens de programação é se seus códigos são compilados para código de máquina ou se são simplesmente 
interpretados. Enquanto no primeiro caso há a figura de um compilador, que transforma o código fonte em código de máquina 
para ele poder então ser executado, no segundo caso,
há a figura de um interpretador que não converte o programa para 
código de máquina, mas sim o interpreta diretamente. 

Linguagens interpretadas possuem vantagens e desvantagens sobre as compiladas: embora elas sejam mais fáceis 
de serem multiplataforma (basta o interpretador
estar disponível para aquela plataforma) e permitam escopo e tipagem dinâmica, também costumam ser menos eficientes que linguagens compiladas e 
a presença de um interpretador é obrigatória para sua execução. 

Dentre as linguagens interpretadas, uma que adquiriu destaque na área de estatística e bioinformática é a linguagem \emph{R}, que 
será de fundamental importância para este trabalho. 


\subsubsection{Linguagem R}

A linguagem \emph{R} é uma linguagem interpretada bastante utilizada no desenvolvimento de rotinas de bioinformática e na análise de
dados. Muitas funções normalmente utilizadas em análises de dados vêm incluídas na linguagem como por exemplo cálculo da
 média, desvio padrão, ajuste de curva, entre outras. A geração de gráficos do \emph{R} também é bastante elaborada e é possível 
gerar gráficos dos mais diversos tipos. Também há implementações de algoritmos mais elaborados de estatística como por 
exemplo algoritmos de \emph{clustering} que também são bastante usados em bioinformática.

Outro ponto interessante da linguagem \emph{R} são os pacotes extras que se pode baixar. Ao todo, no repositório oficial,
são $2076$ pacotes, para os mais diversos propósitos que podem ser desde funções mais específicas para análise como 
por exemplo o \emph{Bayesclust}\footnote{Disponível em \url{http://cran.r-project.org/web/packages/bayesclust/index.html}}, 
como bibliotecas para geração de gráficos específicos ou utilizando alguma biblioteca externa. como por exemplo o 
\emph{rggobi}\footnote{Disponível em \url{http://cran.r-project.org/web/packages/rggobi/index.html}}. 

Com maior destaque para área de bioinformática, o artigo 
\cite{bioconductor} fala sobre o \emph{Bioconductor}, um projeto  que propõe o fornecimento 
de ferramentas para a análise de dados desta área. O projeto foi iniciado em 2001 e a grande maioria de 
seus pacotes são para o ambiente de programação da linguagem \emph{R}.

\subsection{Linguagens interpretadas e computação distribuída}

Embora já existam ótimas soluções para a execução distribuída de programas compilados, como por exemplo o
 MPI\footnote{disponível em \url{http://www.mcs.anl.gov/research/projects/mpi/}}, não se fala muito em soluções 
para execução distribuída de programas em linguagem interpretada. 
Para a linguagem \emph{R} há um pacote chamado \emph{gridR} que permite o uso do R com o middleware \emph{Condor}, %Colocar note
que é uma solução bastante conhecida para execução de programas em grades.  
Um outro trabalho que relaciona o R com computação distribuída é o \cite{Dias} que 
utiliza o middleware \emph{Alchemi}, cuja finalidade é fazer processamento em grade e é baseado na 
tecnologia \textit{.NET} da Microsoft,  e a interface \textit{COM} junto com o pacote do \emph{R}
\emph{RCom}.  

O artigo \cite{boinc} fala sobre a utilização do Middleware de computação voluntária \emph{BOINC} como solução para computação 
em grade na Universidade de Extremadura na Espanha e dentre os programas executados na grade, haviam
programas em R. Porém isso foi somente instalado em redes de computadores cujo sistema
operacional é o Linux. 


\subsection{Solução multiplataforma}

Embora o ambiente \emph{Linux} seja muito utilizado no ambiente acadêmico e em ambientes de desenvolvimento de softwares, é
muito difícil encontrá-lo em um ambiente doméstico ou de trabalho cotidiano. Embora a cada ano o número de usuários deste 
sistema cresça, percebemos que o sistema normalmente utilizado é o \emph{Microsoft Windows}. O artigo \cite{Dias}
dispõe uma solução usando o \emph{middleware} \emph{Alchemi}, baseado na tecnologia da Microsoft \emph{.NET} que é executado
no sistema operacional \emph{Microsoft Windows}. O artigo \cite{boinc} porém utiliza uma rede com computadores rodando \emph{Linux}

Porém como temos muitos computadores com sistemas \emph{Linux} e \emph{Windows} na rede \emph{CEC} do IME-USP,
pensamos que o ideal seria termos uma solução que pudesse utilizar os dois sistemas. Como tanto o \emph{BOINC} como o 
\emph{R} estão disponíveis para os dois sistemas, um dos objetivos deste trabalho é utilizar os dois sistemas na grade.

\subsection{Utilização do \emph{BOINC} com o  \emph{R}}

Embora a \emph{API} do \emph{BOINC} não esteja disponível para ser utilizada em programas escritos na linguagem \emph{R}, 
a utilização destas duas tecnologias juntas mostra algumas vantagens. Como ambas as tecnologias são de código aberto, não
se gasta na compra de licenças. A possibilidade de se executar rotinas inalteradas e de instalar pacotes
do \emph{R} sob demanda utilizando o \emph{wrapper} também foi um fator decisivo na escolha. A 
utilização bem sucedida do processamento de rotinas em \emph{R} na grade descrita no artigo \cite{boinc} também foi um fator 
determinante nesta decisão, além da disponibilidade multiplataforma de ambas as tecnologias.



 



\section{Conceitos e tecnologias utilizadas}

O desenvolvimento do projeto incluiu diversas tecnologias, sendo as principais a linguagem de Programação \emph{R} e o middleware
para computação voluntária \emph{Boinc}. Dentre os conceitos estudados, podemos destacar a computação em grade.  

\subsection{BOINC}

O BOINC, cujo nome é uma sigla para \textit{Berkeley Open Infrastructure for Network Computing}, é um middleware 
para computação em grade e voluntária e foi criado na Universidade de Berkeley, Califórnia, Estados Unidos.

Inicialmente, o projeto consistia em gerenciar o projeto \textit{SETI@HOME} que possuía dois objetivos:

\begin{itemize}
	\item Provar a viabilidade e a praticidade do conceito ``computação em grade distribuída'';
	\item Fazer um trabalho científico útil fazendo uma análise observacional para detectar vida inteligente fora da Terra.
\end{itemize}

O primeiro objetivo foi concluído com sucesso e o resultado é o \textit{BOINC}. O segundo falhou: nenhuma evidência de 
vida inteligente fora da Terra foi encontrada. 

\subsubsection{Funcionamento do BOINC}

Cada unidade de processamento no Boinc é chamada de \emph{workunit} e é constítuida de arquivos executáveis e 
arquivos de entrada. Depois de processado, os arquivos de saída gerados são enviados para o servidor que
normalmente armazena estes arquivos em um banco de dados ou em um arquivo.

Para gerar um workunit são necessários dois arquivos XML, um deles detalhando a entrada e o 
outro detalhando a saída. Para facilitar a escrita do programa, precisamos escrever para cada arquivo um nome lógico 
que ao enviar e receber o cliente renomeia o arquivo. Por exemplo, temos um programa que lê um arquivo chamado 
\verb#input# e escreve no arquivo \verb#output#, para podermos ter muitos arquivos de entrada com nomes diferentes, quando
criamos uma \emph{workunit}, o servidor coloca um nome único e semelhante ao da workunit nos arquivos de entrada e saída que serão renomeados
pelo cliente para o nome lógico.

O processamento é realizado pelo cliente: o arquivo binário é executado e enquanto ele é executado há um checkpoint
que permite em caso de interrupções retomar o processamento de um determinado ponto. Finalizado o processamento, 
na próxima atualização o cliente avisará ao servidor que o processamento foi finalizado. Um diagrama do funcionamento pode
ser visto na figura \ref{funcionamento-boinc}. 


\begin{figure}[!h]
  \includegraphics[scale=0.5]{boinc-schema.png}
  \caption{Funcionamento do Boinc}
  \label{funcionamento-boinc}
\end{figure}


\subsubsection{Wrapper}


O \emph{Wrapper} é um programa escrito utilizando a \emph{api} do \emph{BOINC}, cujo objetivo é executar aplicações legadas, 
i.e. aplicações que não utilizam a API do \emph{BOINC}, utilizando o \textit{BOINC}. Há uma versão do Wrapper distribuída junto com o 
\textit{BOINC} que utiliza um arquivo XML, mas existe uma outra opção descrita no artigo \cite{hungaro}
que utiliza um shell para a execução dos aplicativos.

O arquivo XML de execução tem a seguinte estrutura:

\begin{verbatim}
<job_desc>
    <task>
        <application>foobar</application>
        [ <stdin_filename>stdin_file</stdin_filename> ]
        [ <stdout_filename>stdout_file</stdout_filename> ]
        [ <stderr_filename>stderr_file</stderr_filename> ]
        [ <command_line>--foo bar</command_line> ]
    </task>
    [ ... ]
</job_desc>
\end{verbatim}

Neste XML, o único campo obrigatório é o \emph{application}, que é a aplicação
que será executada e pode ser distribuída junto com a aplicação ou já existir no 
computador que o cliente estará instalado (para este segundo caso é necessário
informar o caminho inteiro do executável). É possível ter mais de uma tag
task, e o wrapper as executará sequencialmente. É de responsabilidade
do \textit{Wrapper} 

\subsection{R}

A linguagem de programação estatística $R$ é uma implementação da linguagem $S$ e foi criada por John Ihaka e Robert
Gentleman na Universidade de Auckland, da Nova Zelândia. Hoje, a linguagem $R$ é uma linguagem considerada padrão
entre estatísticos no desenvolvimento de softwares estatísticos e na análise de dados. Há também um ambiente 
onde podemos utilizar a linguagem em um interpretador e gerar gráficos de alta qualidade. 

Por padrão, as distribuições estatísticas mais populares podem ser utilizadas e é muito mais simples que em 
linguagens como C, Java a manipulação de matrizes e tabelas de dados.Outro ponto importante na linguagem $R$ 
é a extensibilidade: é muito simples instalar bibliotecas. Hoje em dia, há
cerca de $2000$ bibliotecas no repositório principal (chamado de \emph{CRAN}, 
sigla de \textit{Comprehensible R Archive Network}) com as mais diversas funções 
(desde bibliotecas de gráficos específicos até métodos estatísticos mais complexos). 
Outro repositório bastante utilizado é o Bioconductor, que provê rotinas bastante utilizadas para o processamento
de dados da área de bioinformática.






\section{Atividades realizadas}


O início do projeto deu-se ainda em 2008, com a visita ao colégio Rainha da Paz na Lapa, onde o aluno de mestrado do \textit{IME}
Rodrigo Assirati Dias mantém uma grade de computadores com o middleware \textit{Alchemi} citada no trabalho \cite{Dias}.
Nesta visita foi possível esclarecer dúvidas, entender o funcionamento da grade e receber algumas dicas quanto à manutenção da grade. 
Após esse encontro, começou-se a buscar alternativas para o a computação de alta performance com o \textit{R}. Um primeiro pacote encontrado
que fazia esta função foi o \textit{GridR}, que permite submeter rotinas do \textit{R} para \textit{clusters}, máquinas remotas e 
grades. Um dos arcabouços possíveis para o uso deste pacote é o Condor\footnote{Disponível em \url{http://www.cs.wisc.edu/condor/}}
, desenvolvido pela \textit{University of Wisconsin-Madison} e é bastante utilizado em empresas
de grande porte como a \textit{NASA} e pode ser executado tanto em sistemas
baseados em \textit{UNIX}, como em sistemas \textit{Windows}. Seguindo esta busca, encontramos o 
\textit{middleware} \textit{BOINC}, no artigo \cite{boinc}
sendo utilizado para um propósito semelhante em um trabalho na Universidade de Extremadura, na Espanha 
e decidimos que a abordagem seria interessante para nosso trabalho.

Escolhido o \textit{middleware} nos focamos na instalação do servidor. A própria página do \textit{BOINC} 
possui um guia de instalação do servidor do \textit{middleware} no sistema \verb#Debian GNU\Linux# e por
esta distribuição \textit{Linux} ser bastante conhecida por sua estabilidade, foi instalado este sistema no servidor.
Instalado o servidor, o foco foi em ter uma aplicação em R executando remotamente em uma grade de computadores. 
Este processo no sistema Linux foi relativamente simples: utilizando o ``truque do \textit{shebang}'' é possível 
colocar um script como executável e o \textit{wrapper} executá-lo como se fosse um arquivo compilado. Já para
o sistema Windows %Colocar marca registrada
o trabalho foi mais complicado: havia um bug nas configurações de compilação do \textit{wrapper} e até perceber isso
atrasou bastante o andamento do projeto. Passado isso, foi necessário utilizar um programa escrito em C, que apenas executava 
o interpretador junto com o arquivo com a rotina em \textit{R}.

Finalizado esta parte, nos focamos na aplicação a ser executada na grade. Para isso foi criado um programa em \textit{R} que
apenas acessava um arquivo e fazia alguns cálculos custosos. Isto foi então configurado para o mesmo programa poder
ser executado tanto em sistemas Windows como em sistemas Linux. Paralelamente a isso, foi pedido para a administração
da rede do \textit{CEC} para instalar o \textit{BOINC} nas máquinas da rede, o qual foi realizado em janeiro de $2010$. 
No momento, $69$ máquinas possuem fazem parte do grade. Este número dificilmente aumentará, já que as máquinas Windows
que terão o \emph{Boinc} instalado já possuem o sistema Linux em outra partição.





\section{Resultados e produtos obtidos}


O principal resultado deste trabalho foi fazer o \emph{Boinc} funcionar com rotinas em \emph{R}, tanto
nos ambientes \emph{Linux} como no ambiente \emph{Windows} % colocar marca registrada

\subsection{Implementação}

\subsubsection{Sistema Linux}

Para o ambiente \emph{Linux}, foi relativamente simples o processo: dado um arquivo com rotinas
do \emph{R} a serem executadas, é somente necessário alterar a permissão do arquivo para executável 
(via \verb#chmod +x arquivo.R#) e adicionar a seguinte linha no início do arquivo:

\begin{verbatim}
#!/usr/bin/Rscript
\end{verbatim}

Isso faz um sistema \emph{Linux} chamar o interpretador \emph{Rscript} para interpretar o arquivo  
e assim fazer a interpretação do arquivo. Esta solução permite não só que rotinas em \emph{R} sejam
executadas, mas sim qualquer script que tenha seu interpretador descrito na primeira linha.

A esta solução, demos o nome de \emph{Truque do Shebang}, pelos caracteres \verb'#!' serem chamados
popularmente de \emph{shebang}. Esta implementação foi relativamente simples: o BOINC possuía um tutorial
para executarmos programas compilados na grade e só o truque do shebang foi necessário.

Porém, para termos a mesma solução em ambos os sistemas, utilizamos a solução para
Windows no sistema Linux. 

\subsubsection{Sistema Windows} %Marca registrada

Como o sistema Windows não permite utilizar script utilizando o \emph{shebang}, foi necessário 
utilizar um programa compilado escrito na linguagem C, que chamamos de \emph{Runner}, 
que usando a função \verb#system#, chama o interpretador com o arquivo.R como parâmetro. 

Esta solução permite inclusive que usemos scripts diferentes de R a cada vez que criamos um 
\emph{workunit}, assim como adicionar arquivos extra que por ventura fossem necessários para
o processamento. Esta maneira também funciona no \emph{Linux}, só que o \emph{Runner} precisar
ser compilado para o \emph{Linux} com o caminho para o interpretador correto. O programa também não faz
uma verificação para perceber se o \emph{R} está instalado. 

A utilização do \emph{Runner} foi necessário devido ao \emph{Wrapper} não perceber corretamente
que o interpretador foi executado sem erros e, mesmo em execuções sem erros, o \emph{wrapper} 
recebia um valor de retorno do interpretador diferente de zero, o que ele percebia como um erro
e marcava o \emph{workunit} como inválido. Um diagrama exemplificando esse funcionamento pode ser visto 
na figura \ref{diagrama-boinc-wrapper-windows}. Para a execução ficar multiplataforma, foi necessário 
fazer uma pequena alteração no \emph{wrapper} na versão para \emph{Linux} modificando o nome do
arquivo XML do wrapper, já que em ambos os sistemas é necessária a utilização de um arquivo XML próprio.


A implementação no Windows foi bastante difícil: existem muitos pequenos detalhes que acabam
atrapalhando um pouco. Um dos exemplos foi especificar o caminho do executável: para o Windows acessar
um diretório específico devemos ``truncar'' os nomes de diretórios maiores que $8$ caracteres, sendo 
assim apontamos o caminho, trocando o oitavo e nono caractere do diretório por \verb#~1# 
e desprezando os seguintes. 
Alguns outros problemas foram relacionados à falta de mensagens de erro do BOINC mais amigáveis e
à falta de verificação das entradas: um \verb#XML# mal formatado causou um erro  no processamento
do workunit, quando isso deveria ser acusado na submissão do workunit. Outro problema foi um \emph{Bug}
nas configurações de compilação do \emph{wrapper} no Microsoft Visual Studio Express % Verificar 
que fazia referência a bibliotecas desnecessárias e inexistentes para a compilação do \emph{wrapper}
e isso só foi resolvido pelos desenvolvedores do \emph{BOINC} algumas semanas depois. 


\begin{figure}[!h]
  \centering
  \includegraphics[scale=0.3]{boinc-diagram-runner.png}
  \caption{Diagrama do funcionamento do \emph{BOINC} com o \emph{Runner} e \emph{Wrapper}}
  \label{diagrama-boinc-wrapper-windows}
\end{figure}
 

\subsection{Discussão}



\subsubsection{Vantagens} 

As principais vantagens no uso do \emph{Boinc} para o processamento em grade são:

\begin{itemize}
  \item \textbf{Facilidade de se adicionar novos nós} - A instalação do BOINC em ambos os sistemas Linux e Windows é simples
e fácil de ser feita e nenhuma ação no servidor é necessária a cada instalação de nós. Além disso, é muito simples fazer
a replicação de configurações, tanto para o processamento, quanto para a conexão com o servidor para os computadores;
  \item \textbf{Processamento multiplataforma} - Para a grade funcionar na plataforma só são necessárias duas coisas: 
que o BOINC e o \emph{R} estejam disponível para a plataforma. As plataformas mais comuns 
(Linux $32$ e $64$ bits e Microsoft Windows) têm ambos os projetos disponíveis;
  \item \textbf{Código aberto} - A utilização de dois softwares com código aberto facilita bastante: a 
busca de bugs se torna possível, a gratuidade dos softwares e a grande quantidade de documentação, muitas vezes produzidas por
pessoas que não necessariamente são da equipe de desenvolvimento do \emph{BOINC}. A mentalidade de ajuda mútua, existente nas
listas de discussão e no canal de IRC do projeto também é de grande ajuda; 
  \item \textbf{Execução invisível ao usuário} - Com o \emph{BOINC} configurado para isso, um usuário comum da rede nem ao menos 
percebe a existência de um processamento em grade. É possível configurar o \emph{BOINC} para só começar a execução com o computador ocioso
após um número arbitrário de minutos. Também é possível configurar para o processamento só acontecer em determinadas 
faixas de horários e dias da semana. Outra configuração interessante é a determinação do máximo de memória 
\emph{RAM} e de espaço em disco para a execução, assim como a frequência com que ele usará a rede. 
  \item \textbf{Solução funciona para qualquer linguagem de script} - De maneira análoga, é possível executar qualquer programa escrito em 
linguagem interpretada com o BOINC utilizando esta mesma solução. Como comentado antes, só é necessário que exista uma versão do interpretador
para as plataformas necessárias (o que é comum para as linguagens mais utilizadas como \emph{PERL}, \emph{Python} e 
\emph{Ruby} e as plataformas mais comuns). 

\end{itemize}

\subsubsection{Desvantagens}

As principais desvantagens são:

\begin{itemize}
  \item \textbf{Necessidade de se ter o \emph{R} instalado} - O \emph{R} não é uma linguagem instalada por padrão
nas distribuições Linux mais populares, nem no \emph{Windows}. Então, a adição de um nó só pode ser feita se o \emph{R}
for também instalado. 
  \item \textbf{Falta de checkpoints} - Utilizando um aplicativo feito com a \emph{API} do \emph{BOINC} é possível se ter
\emph{Checkpoints}, que são uma maneira de uma aplicação feita com o \emph{BOINC} reiniciar o processamento
não do início, mas sim de um determinado ponto. Utilizando o \emph{Wrapper} e o \emph{R}, perdemos esse recurso. A computação
de rotinas longas se torna mais difícil e pouco recomendada, já que a cada interrupção o processamento é reiniciado. 
  \item \textbf{Falta de ``compromisso'' fixo dos clientes} - Diferentemente da rede citada no artigo \cite{Dias}
não há a figura de um computador \emph{Manager} que gerencia as máquinas, atualizando a qualquer momento, mas sim um servidor 
que apenas envia e recebe as tarefas e a iniciativa de computação fica com os computadores da grade. 

\end{itemize}

\subsection{Instalação da rede}

A instalação para o sistema Linux foi concluída no laboratório CEC do IME-USP. No momento possuímos $69$ máquinas com o sistema 
Linux funcionando plenamente e 2 máquinas com algum problema na instalação dos pacotes 
necessários para a computação que foram descartadas dos testes. 

No momento está sendo feito a instalação no sistema Windows que contará com $16$ máquinas que possuem um dual-boot
com máquinas com o sistema Linux já presentes na grade. Esta instalação se mostrou um pouco mais complexa, devido a configurações que devem ser feitas no momento da instalação. 


\subsection{Testes}

Para testar a grade foi criado um script na linguagem \emph{R} que apenas fazia contas sem significado 
por, aproximadamente, $4min20s$ ``dedicados'' (isto é, executados diretamente em um dos computadores da grade). Foram criados $150$ \textit{workunits} para serem distribuídos entre as máquinas  processadas. Como
configurado no servidor, foi utilizada uma redundância que enviava, para cada \textit{workunit}, duas tarefas a serem processadas.

Devido a utilização do CEC no mês de janeiro e início de feveiro para cursos de verão (no qual um deles utilizou o sistema em uma das salas Windows), somente duas máquinas fizeram o processamento das $300$ tarefas. O tempo médio 
de processamento das tarefas foi de $9min2.167s$, com um desvio padrão de $14.486s$. Um histograma desta execução pode 
ser encontrado na figura \ref{histograma}. Um dos computadores teve um desempenho pouco superior ao outro tendo uma
média de $8min50.130s$ (e um desvio padrão de $6.543s$) contra uma média de $9min16.404s$ (com um desvio padrão de $5.544s$) do outro computador. O computador que foi mais rápido processou $154$ tarefas, contra $146$ do outro. Todas
execuções foram feitas com sucesso e todos os resultados gerados estão presentes no diretório de upload dos resultados. 

O tempo total de processamento das tarefas foi de $14h31min41s$. Caso as tarefas fossem processadas
em série, cada execução durasse o tempo de execução direta, em um computador e de maneira direta 
(sem o \emph{BOINC}) o tempo de execução seria de, aproximadamente, $21h40min$ o que nos dá um \emph{speed-up} 
próximo de $1.5$.













\section{Conclusões}

\bibliographystyle{amsalpha}
\bibliography{bibliografia}


\newpage


A principal conclusão deste trabalho é que é possível a utilização do \emph{BOINC} para o processamento
de rotinas de bioinformática escritas na linguagem \emph{R}. A utilização multiplataforma
também foi de grande utilidade já que podemos incluir praticamente qualquer tipo de computador 
existente em redes. O custo total de implementação foi somente a inutilização de uma máquina, utilizada
como servidor da rede. Embora o \emph{BOINC} seja um projeto consolidado e possua uma documentação com bastante conteúdo,
as mensagens de erro, principalmente do \emph{Wrapper}, são muito pouco informativas, o que tornou o desenvolvimento 
deste projeto bastante trabalhoso. 

A instalação do \emph{BOINC} e do \emph{R} na rede do CEC também foi bastante simples pois estes estão disponíveis
como pacotes do Debian (que é a distribuição Linux que é executada nos laboratórios do CEC). O \emph{deploy} da aplicação
de teste também foi simples, já que o BOINC possui um utilitário de linha de comando e com apenas um comando (apontando 
o servidor e a chave de acesso) o deploy da aplicação é feito.

Para a continuação do projeto há várias sugestões:

\begin{itemize}
  \item \emph{Benchmark} da rede e comparação com grade descrita no artigo \cite{Dias}
Para determinar a viabilidade, seria interessante estabelecermos a comparação com outra 
alternativa. 
  \item Comparação com grades ``alugadas'': hoje já existe oportunidade de se fazer esse tipo de processamento.
em grades alugadas como a oferecida pela empresa \emph{Amazon}. Como os computadores na rede 
consomem energia elétrica seria interessante comparar o gasto da energia elétrica com o gasto em 
uma grade ``alugada''.
  \item Analisar o desempenho nas máquinas com Windows e com Linux. Seria interessante analisar 
o Benchmark da grade em ambos os sistemas e determinar qual das duas plataformas é mais
propícia para o processamento;
  \item Utilização de máquinas virtuais: como feito no artigo \cite{boinc}, podemos utilizar
máquinas virtuais, que são iniciadas em cada nó e é feito o processamento. Sem ter que se preocupar
com a instalação do \emph{R} em todas as máquinas 
\end{itemize}




\part{Parte subjetiva}

\section{Desafios e frustrações encontrados}

O curso de bacharelado em ciência da computação é um curso bastante denso e dificilmente temos tempo
para fazer todas as tarefas de todas as disciplinas. Então acredito que o meu maior desafio nestes anos de IME 
foi conciliar todas as tarefas e disciplinas, e infelizmente descartando alguma as vezes. 

\section{Disciplinas relevantes para o trabalho}

Diversas disciplinas foram relevantes para este trabalho:

\begin{itemize}
  \item \textbf{MAC110 - }
  \item \textbf{MAC122 - }
  \item \textbf{MAC211 - Laboratório de programação I} Ferramentas como versionamento de código, processamento de 
texto e o makefile foram essenciais para a elaboração deste trabalho de forma indireta e contribuiram com a 
boa qualidade do mesmo.
  \item \textbf{MAC242 - Laboratório de programação II} Linguagens de script facilitam bastante o trabalho de tarefas 
repetitivas e o boa parte do que sei sobre este tipo de linguagem eu aprendi neste curso.
  \item \textbf{MAC338 - Análise de algoritmos} Este curso contribuiu indiretamente com minha formação como cientista da computação i
e muitos dos conceitos aprendidos neste curso ajudaram o entendimento melhor de algoritmos e soluções.
  \item \textbf{Programação paralela e distribuída}
  \item \textbf{MAC422 - Sistemas Operacionais} Saber como um programa é executado, como o sistema gerencia essas execução e como a tabela de processos
funciona é essencial quando se trabalha com uma grade de computadores.  
\end{itemize}

% desafios e frustrações encontrados;
% lista das disciplinas cursadas no BCC mais relevantes para o trabalho;
% observações sobre a aplicação de conceitos estudados nas disciplinas do curso;
% se o aluno fosse continuar atuando na área em que realizou o trabalho, que passos tomaria para aprimorar os conhecimentos relevantes para esta atividade?

\section{Desafios e frustrações encontrados}

O curso de bacharelado em ciência da computação é um curso bastante denso e dificilmente temos tempo
para fazer todas as tarefas de todas as disciplinas. Então acredito que o meu maior desafio nestes anos de IME 
foi conciliar todas as tarefas e disciplinas, e infelizmente descartando alguma em alguns momentos 
(principalmente finais de semestre). 

A falta de aprendizado de orientação a objeto também foi uma frustração: como este assunto é muito requisitado por empresas
e projetos, não aprendê-lo foi bastante frustrante. Há também pouco enfoque a tecnologias, deveriam haver mais chances de
aprendermos tecnologias e desenvolvermos mais projetos. 

Para algumas disciplinas também que pedem projetos e trabalhos únicos que são entregues no decorrer do semestre, 
falta uma coisa muito importante: um \emph{Feedback} constante do professor/monitor para saber se estamos no caminho certo. 
No decorrer do curso conclui algumas matérias com trabalhos que eu acreditei estar bom, mas com uma nota abaixo do esperado. 
Fazer um trabalho de conclusão de curso também é uma escolha muito feliz: serve para coroarmos nosso aprendizado
e aplicá-los utilizando os temas que mais gostamos.  

A figura de um orientador e de um tutor também foi de suma importância: é sempre interessante ter alguém que se possa receber
conselhos de matérias e de optativas. Desenvolver também um projeto de iniciação científica também foi importante, para 
meu desenvolvimento e acredito que para qualquer aluno seria uma experiência válida. 

 

\section{Disciplinas relevantes para o trabalho}

Diversas disciplinas foram relevantes para este trabalho:

\begin{itemize}
  \item \textbf{MAC122 - Principio e desenvolvimento de algoritmos } - Este curso fornece uma base importante para as outras matérias e
ajuda a melhorar o raciocínio para elaborarmos algoritmos.
  \item \textbf{MAC323 - Estruturas de Dados } - Aprender as diversas estruturas de dados, seu uso e a eficiência é essencial
para qualquer trabalho que envolva ciência da computação.
  \item \textbf{MAC211 - Laboratório de programação I} - Ferramentas como ver\-si\-o\-na\-men\-to de código, processamento de 
texto e o makefile foram essenciais para a elaboração deste trabalho de forma indireta e contribuíram com a 
boa qualidade do mesmo.
  \item \textbf{MAC242 - Laboratório de programação II} - Linguagens de script facilitam bastante o trabalho de tarefas 
repetitivas e o boa parte do que sei sobre este tipo de linguagem eu aprendi neste curso.
  \item \textbf{MAC338 - Análise de algoritmos} - Este curso contribuiu indiretamente com minha formação como cientista da computação i
e muitos dos conceitos aprendidos neste curso ajudaram o entendimento melhor de algoritmos e soluções.
  \item \textbf{MAC431 - Programação paralela e distribuída} Aprender a utilidade de programação de alta performance foi de bastante utilidade
neste trabalho.
  \item \textbf{MAC422 - Sistemas Operacionais} - Saber como um programa é executado, como o sistema gerencia essas execução e como a tabela de processos
funciona é essencial quando se trabalha com uma grade de computadores. 
  \item \textbf{IBI5031 - Reconhecimento de padrões} - Neste curso tive a oportunidade de aprender algoritmos e fazer análises de dados, 
que são normalmente feitas em pesquisas de bioinformática. Também programei bastante na linguagem \emph{R} neste curso e percebi sua
importância
\end{itemize}

\section{Conceitos Aplicados}

Dentre os conceitos aprendidos no curso utilizados no trabalho pode-se citar a computação distribuída, aprendida no curso de
computação paralela e distribuída. Saber que alguns problemas só podem ser resolvidos de uma exaustiva, também 
foi importante e esse foi um conceito aprendido no curso de Análise de Algoritmos.  


\section{Continuação do trabalho}

A continuação do trabalho será escrever um artigo falando sobre a grade para ser publicado na página do \emph{BOINC} e em 
alguma publicação. Outro trabalho interessante decorrente deste 
seria melhorar o \emph{Wrapper} e o \emph{BOINC} em geral para fazê-los terem mensagens de erros mais claras. 

\section{Agradecimentos}

Este trabalho foi feito com a ajuda de diversas pessoas. Indiretamente, meus pais, familiares, amigos e namorada foram de extrema importância 
no apoio, amizade e carinho. 

Diretamente, agradeço à CNPq pela bolsa que me permitiu a dedicação exclusiva a este projeto. Agradeço também ao meu orientador, 
Prof. Dr. Roberto Hirata Jr. que me apoiou e me ajudou bastante na elaboração deste trabalho. Para a obtenção do servidor,
agradeço ao Prof. Dr. Roberto Marcondes César Junior, e aos administradores da Rede Vision, David Pires e Rodrigo Bernardo Pimentel, 
que me ajudaram quando na substituição do servidor com um problema de \emph{hardware}. Agradeço também ao aluno de pós-graduação
Rodrigo Assirati Dias, que me ajudou e me deu diversas dicas com relação à manutenção e a instalação da grade. Na parte do \emph{BOINC}
agradeço à Nicolás Alvarez e Yoyo, que me ajudaram a resolver os bugs e a implementar a solução descrita pelo trabalho. Agradeço também 
o Prof. Dr. Alfredo Goldman pela disponibilização da rede \emph{CEC} e aos administradores da rede, Jânio Matsuura e Gislaine Olivi
pela instalação da grade.






% desafios e frustrações encontrados;
% lista das disciplinas cursadas no BCC mais relevantes para o trabalho;
% observações sobre a aplicação de conceitos estudados nas disciplinas do curso;
% se o aluno fosse continuar atuando na área em que realizou o trabalho, que passos tomaria para aprimorar os conhecimentos relevantes para esta atividade?
\end{document}
