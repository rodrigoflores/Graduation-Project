% desafios e frustrações encontrados;
% lista das disciplinas cursadas no BCC mais relevantes para o trabalho;
% observações sobre a aplicação de conceitos estudados nas disciplinas do curso;
% se o aluno fosse continuar atuando na área em que realizou o trabalho, que passos tomaria para aprimorar os conhecimentos relevantes para esta atividade?

\section{Desafios e frustrações encontrados}

O curso de bacharelado em ciência da computação é um curso bastante denso e dificilmente temos tempo
para fazer todas as tarefas de todas as disciplinas. Então acredito que o meu maior desafio nestes anos de IME 
foi conciliar todas as tarefas e disciplinas, e infelizmente descartando alguma as vezes. 

A falta de aprendizado de orientação a objeto também foi uma frustração: como este assunto é muito requisitado por empresas
e projetos, não aprendê-lo foi bastante frustrante. Há também pouco enfoque a tecnologias, deveriam haver mais chances de
aprendermos tecnologias e desenvolvermos mais projetos. 

Para algumas disciplinas também que pedem projetos e trabalhos únicos que são entregues no decorrer do semestre, 
falta uma coisa muito importante: um \emph{Feedback} constante do professor/monitor para saber se estamos no caminho certo. 
No decorrer do curso conclui algumas matérias com trabalhos que eu acreditei estar bom, mas com uma nota abaixo do esperado. 
Fazer um trabalho de conclusão de curso também é uma escolha muito feliz: serve para coroarmos nosso aprendizado
e aplicá-los utilizando os temas que mais gostamos.  

A figura de um orientador e de um tutor também foi de suma importância: é sempre interessante ter alguém que se possa receber
conselhos de matérias e de optativas. Desenvolver também um projeto de iniciação científica também foi importante, para 
meu desenvolvimento e acredito que para qualquer aluno seria uma experiência válida. 

 

\section{Disciplinas relevantes para o trabalho}

Diversas disciplinas foram relevantes para este trabalho:

\begin{itemize}
  \item \textbf{MAC122 - Principio e desenvolvimento de algoritmos } - Este curso fornece uma base importante para as outras matérias e
ajuda a melhorar o raciocínio para elaborarmos algoritmos.
  \item \textbf{MAC323 - Estruturas de Dados } - Aprender as diversas estruturas de dados, seu uso e a eficiência é essencial
para qualquer trabalho que envolva ciência da computação.
  \item \textbf{MAC211 - Laboratório de programação I} - Ferramentas como versionamento de código, processamento de 
texto e o makefile foram essenciais para a elaboração deste trabalho de forma indireta e contribuíram com a 
boa qualidade do mesmo.
  \item \textbf{MAC242 - Laboratório de programação II} - Linguagens de script facilitam bastante o trabalho de tarefas 
repetitivas e o boa parte do que sei sobre este tipo de linguagem eu aprendi neste curso.
  \item \textbf{MAC338 - Análise de algoritmos} - Este curso contribuiu indiretamente com minha formação como cientista da computação i
e muitos dos conceitos aprendidos neste curso ajudaram o entendimento melhor de algoritmos e soluções.
  \item \textbf{MAC431 - Programação paralela e distribuída} Aprender a utilidade de programação de alta performance foi de bastante utilidade
neste trabalho.
  \item \textbf{MAC422 - Sistemas Operacionais} - Saber como um programa é executado, como o sistema gerencia essas execução e como a tabela de processos
funciona é essencial quando se trabalha com uma grade de computadores. 
  \item \textbf{IBI5031 - Reconhecimento de padrões} - Neste curso tive a oportunidade de aprender algoritmos e fazer análises de dados, 
que são normalmente feitas em pesquisas de bioinformática. Também programei bastante na linguagem \emph{R} neste curso e percebi sua
importância
\end{itemize}

\section{Conceitos Aplicados}

Dentre os conceitos aprendidos no curso utilizados no trabalho pode-se citar a computação distribuída, aprendida no curso de
computação paralela e distribuída. Saber que alguns problemas só podem ser resolvidos de uma exaustiva, também 
foi importante e esse foi um conceito aprendido no curso de Análise de Algoritmos.  


\section{Continuação do trabalho}

A continuação do trabalho será escrever um artigo falando sobre a grade para ser publicado na página do \emph{BOINC} e em 
alguma publicação. Outro trabalho interessante decorrente deste 
seria melhorar o \emph{Wrapper} e o \emph{BOINC} em geral para fazê-los terem mensagens de erros mais claras. 

\section{Agradecimentos}

Este trabalho foi feito com a ajuda de diversas pessoas. Indiretamente, meus pais, familiares, amigos e namorada foram de extrema importância 
no apoio, amizade e carinho. 

Diretamente, agradeço à CNPq pela bolsa que me permitiu a dedicação exclusiva a este projeto. Agradeço também ao meu orientador, 
Prof. Dr. Roberto Hirata Jr. que me apoiou e me ajudou bastante na elaboração deste trabalho. Para a obtenção do servidor,
agradeço ao Prof. Dr. Roberto Marcondes César Junior, e aos administradores da Rede Vision, David Pires e Rodrigo Bernardo Pimentel, 
que me ajudaram quando na substituição do servidor com um problema de \emph{hardware}. Agradeço também ao aluno de pós-graduação
Rodrigo Assirati Dias, que me ajudou e me deu diversas dicas com relação à manutenção e a instalação da grade. Na parte do \emph{BOINC}
agradeço à Nicolás Alvarez e Yoyo, que me ajudaram a resolver os bugs e a implementar a solução descrita pelo trabalho. Agradeço também 
o Prof. Dr. Alfredo Goldman pela disponibilização da rede \emph{CEC} e aos administradores da rede, Jânio Matsuura e Gislaine Olivi
pela instalação da grade.



