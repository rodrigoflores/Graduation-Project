\documentclass[a4paper,12pt,titlepage]{article}
\usepackage[T1]{fontenc}
\usepackage[brazil]{babel}
\usepackage[utf8x]{inputenc}
\usepackage{graphics}
\usepackage{times}
\usepackage{ucs}
\usepackage{url}



\title{Trabalho de Conclusão de Curso \\ }


\author{Rodrigo L. M. Flores \and
        Orientador: Roberto Hirata Jr. }


\date{\today}

\begin{document}

\maketitle

\section{Introdução}

O ponto de partida para este projeto vem da necessidade de se fazer processamentos custosos de 
dados em bioinformática. Estes processamentos costumam ser combinatórios, o que os fazem
demorar um bom tempo sendo executados em um computador comum, sendo necessário utilizar recursos
computacionais de alta performance para obter os resultados em tempo hábil.

\subsection{Programação paralela e distribuída}

Uma abordagem clássica para se resolver esse problema é utilizar \textit{clusters} que são um grupo de computadores
ligados entre si de modo a parecer ser um único computador muito mais potente. \textit{Clusters} podem ser tanto máquinas
específicas para isso, produzidas com um alto custo e com um hardware específico para otimizar seu desempenho, ou 
pode ser utilizado o conceito de computação em grades: utilizar computadores comuns trabalhando em paralelo 
para fazer o processamento. 

Uma outra alternativa é utilizar grades computacionais distribuídas. Um dos projetos mais famosos neste 
campo é o \textit{SETI@home}, cujos objetivo originais eram:

\begin{itemize}
  \itemize Provar a viabilidade e a praticidade de grades computacionais distribuídas;
  \itemize Fazer um trabalho útil e apoiar uma análise de observações para detectar vida inteligente fora da Terra.
\end{itemize}

O primeiro destes dois objetivos foi cumprido com sucesso e um middleware para este fim foi criado, o \textit{BOINC}. Já
o segundo não obteve sucesso: não foi encontrado sinais de vida inteligente fora da Terra. 





\end{document}
