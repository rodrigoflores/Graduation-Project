\documentclass[a4paper,12pt,titlepage]{article}
\usepackage[T1]{fontenc}
\usepackage[brazil]{babel}
\usepackage[utf8x]{inputenc}
\usepackage{graphics}
\usepackage{times}
\usepackage{ucs}
\usepackage{url}



\title{Trabalho de Conclusão de Curso \\ }


\author{Rodrigo L. M. Flores \and
        Orientador: Roberto Hirata Jr. }


\date{\today}

\begin{document}

\maketitle

\section{Introdução}

O ponto de partida para este projeto vem da necessidade de se fazer processamentos custosos de 
dados em bioinformática. Estes processamentos costumam ser combinatórios, o que os fazem
demorar um bom tempo sendo executados em um computador comum, sendo necessário utilizar recursos
computacionais de alta performance para obter os resultados em tempo hábil.

\subsection{Programação paralela e distribuída}

Uma abordagem clássica para se resolver esse problema é utilizar \textit{clusters} que são um grupo de computadores
ligados entre si de modo a parecer ser um único computador muito mais potente. \textit{Clusters} podem ser tanto máquinas
específicas para isso, produzidas com um alto custo e com um hardware específico para otimizar seu desempenho, ou 
pode ser utilizado o conceito de computação em grades: utilizar computadores comuns trabalhando em paralelo 
para fazer o processamento. 

Uma outra alternativa é utilizar grades computacionais distribuídas. Grades computacionais costumam ser constituídas
de um conjunto de computadores cujo principal objetivo não é fazer processamento pesado, mas sim lidar com o usuário 
cotidiano que normalmente utiliza a máquina para acessar a internet, ler seu correio eletrônico, elaborar
trabalhos usando processadores de texto e planilhas eletrônicas, entre outras coisas. Embora grades computacionais tenham um 
processamento menor que clusters dedicados, costumam ter um custo muito menor, já que normalmente computadores ``comuns'' já existem em
uma empresa ou universidade. 


\subsection{Computação voluntária}
%Bibliografia Wikipedia


A computação voluntária é um tipo de computação distribuída no qual pessoas que possuem computadores podem doar processamento e
armazenamento ocioso de suas máquinas enquanto elas estiverem ociosas. Estes projetos normalmente têm um objetivo bem definide.

O primeiro projeto de Computação voluntária foi o \textit{Great Internet Mersenne Prime Search}, lançado em janeiro de 1996. 
Seu objetivo era encontrar números primos de Mersenne\footnote{Isto é, números primos na forma $M_n = 2^n - 1$}. Em seguida houveram 
muitos outros projetos e alguns utilizam um apelo social, de modo a obter mais doadores de processamento, um exemplo disso é o
o Folding At Home, que investiga o enrolamento de proteínas e que pode ajudar o desenvolvimentos de pesquisas contra 
Câncer, doença de Huntington, entre outras. 

Um dos projetos mais notáveis de computação voluntária foi o \textit{SETI@Home}. Os objetivos iniciais deste projeto eram: 

\begin{itemize}
  \item Provar a viabilidade e a praticidade de grades computacionais distribuídas;
  \item Fazer um trabalho útil e apoiar uma análise de observações para detectar vida inteligente fora da Terra.
\end{itemize}

Este projeto atraiu centenas de milhares de voluntários, porém só o primeiro objetivo teve sucesso: não foram encontrados sinais 
de vida inteligente fora da Terra. Um middleware então foi criado para este fim: o \textit{BOINC}. 

\subsection{Linguagens Interpretadas}



\end{document}
