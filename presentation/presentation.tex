\documentclass{beamer}
\usepackage[brazil]{babel}
%\usepackage[latin1]{inputenc}
\usepackage[utf8x]{inputenc} 
%\usepackage[all]{xy}
\PrerenderUnicode{ç}

\setbeamercovered{transparent=5}

\usetheme{Antibes}

\title{BOINC + R: Executando rotinas de bioinformática em grades oportunistas}

\author{Rodrigo L. M. Flores \\ \url{flores@ime.usp.br}}


\institute{Instituto de Matemática e Estatística\\Universidade de São Paulo}

\begin{document}

\date{\today}

\frame{\titlepage}

\frame{\tableofcontents}

\section{Introdução}

\begin{frame}
  \frametitle{Bioinformática}
  \begin{itemize}
    \item Aplicação de técnicas computacionais e matemáticas para generação, gerenciamento e análise de bioinformação 
    \item Análise de expressão de genes de dados, Clustering
    \item Algoritmos combinatórios e custosos
  \end{itemize} 
\end{frame}

\begin{frame}
  \frametitle{Computação em grade}
  \begin{itemize}
    \item Clusters dedicados: caros e específicos;
    \item Computadores pessoais em um laboratório ociosos na maior parte do tempo
    \item Não há necessidade de se comprar novos equipamentos
  \end{itemize}
\end{frame}

\begin{frame}
  \frametitle{Computação voluntária}
  \begin{itemize}
    \item Slogan: ``Doe seus ciclos de CPU para um projeto'';
    \item Normalmente projetos com apelo ``humanitário'' (pesquisas de enovelamento de proteínas, malária, descoberta de vida extraterrestre, resolução de instâncias grandes do N-Rainhas, etc);
    \item Computação em grade cuja rede em comum é a Internet; 
    \item Vontade de entrar na grade é do dono do computador;
    \item É possível configurar para que os ciclos de CPU sejam doados se o computador estiver ocioso, se estiver funcionando na energia elétrica, ou programar perfis diferentes em horários ou dias da semana
    \item Exemplos de projetos: SETI@HOME, Folding@Home, N Queens At Home, World Community Grid
    \item Projetos consolidados e funcionando a anos
  \end{itemize}
\end{frame}

\begin{frame}
  \frametitle{Uso de computação voluntária como uma grade}
  \begin{itemize}
    \item Idéia surgiu na Universidade de Extremadura, Espanha
    \item Computadores são de uma empresa ou instituição, e a decisão de como eles devem ser utilizados, não são apenas mas sim de uma instituição
    \item Utilização de uma rede de laboratórios para isto
    \item Possibilidade de executar projetos próprios ou projetos já existentes
  \end{itemize}
\end{frame}


\section{Ferramentas utilizadas}

\begin{frame}
  \frametitle{BOINC}
  %Colocar Logo
  \begin{itemize}
    \item Um produto do projeto SETI@HOME (que infelizmente falhou em achar vida inteligente fora da terra)
    \item Middleware que possui suporte a vários projetos 
    \item Software Livre (tanto no servidor como no cliente)
  \end{itemize}
\end{frame}

\begin{frame}
  \frametitle{R}
  \begin{itemize}
    \item Linguagem de programação interpretada, baseada na linguagem S e com enfoque estatístico
    \item Muito utilizada na área de bioinformática
    \item Possibilidade de se compartilhar módulos (há mais de mil pacotes disponíveis para download) 
  \end{itemize}

\end{frame}

\begin{frame}
  \frametitle{Servidor}
  \begin{itemize}
    \item Hardware: AMD Athlon X2 4600+, 4GB de RAM %Confirmar
    \item Sistema Debian GNU/Linux
    \item BOINC Server versão 6.7.0
  \end{itemize}
\end{frame}

\begin{frame}
  \frametitle{Clientes}
  \begin{itemize}
    \item 1 máquina Windows
    \item 3 máquinas Linux
    \item Em processo de expansão (rede em ``reforma'')
  \end{itemize}
\end{frame}

\begin{frame}
  \frametitle{Aplicações Legadas}
  \begin{itemize}
    \item Aplicações já compiladas
    \item Não há a necessidade de conhecer ou alterar o código
    \item Uso do wrapper
  \end{itemize}
\end{frame}

\begin{frame}[fragile]
  \frametitle{Wrapper}
  \begin{itemize}
    \item Programa feito com a API do Boinc que lê um XML (chamado job.xml)  com informações de como executar um executável
    \item Disponível para Linux e Windows
  \end{itemize}
\end{frame}

\begin{frame}[fragile]
  \frametitle{Job.xml}
  Opções mais utilizadas:
 \begin{verbatim}
 <job_desc>
    <task>
        <application>runner</application>
        (opcional) <stdout_filename>stdout</stdout_filename>
        (opcional) <stderr_filename>stderr</stderr_filename>
        (opcional) <stdin_filename>stdin</stdin_filename> 
        (opcional) <command_line>--help</command_line>
    </task>
  </job_desc>
  \end{verbatim}
\end{frame}



\begin{frame}
  \frametitle{No nosso caso...}
  \begin{itemize}
    \item[Application] Aplicação compilada que possui apenas uma chamada de sistema chamando o interpretador + script
    \item{std\*} Arquivo de entrada (in), saída (out), saída de erro (err)
    \item{command_line} Parâmetros de linha de comando
  \end{itemize}
\end{frame}

\section{Discussão e resultados} 

\begin{frame}
  \frametitle{Sistema Linux}
  \begin{itemize}
    \item Processo simples de compilação (via makefile)
    \item Possibilidade de usar o truque do \textit{shebang}
  \end{itemize}
\end{frame}

\begin{frame}
  \frametitle{Sistema Windows}
  \begin{itemize}
    \item Processo de compilação mais complexo (via Microsoft Visual C++)
    \item Configuração de compilação do wrapper com problema
    \item Dificuldades
  \end{itemize}
\end{frame}

\begin{frame}
  \frametitle{Instalação na rede do CEC}
  \begin{itemize}
    \item Instalação simples no sistema Linux e no Windows
    \item Reforma na rede do CEC
    \item No momento 3 máquinas Linux e 1 máquina Windows instalada
    \item Instalação não finalizada
  \end{itemize}
\end{frame}

\section{Conclusão}

\begin{frame}
  \frametitle{Agradecimentos}
  \begin{itemize}
    \item Prof. Dr. Roberto Hirata Jr.
    \item Prof. Dr. Alfredo Goldman
    \item Rodrigo Assirati Dias
    \item Nicolás Alvarez e Yoyo (desenvolvedores do BOINC)
    \item Gislaine Olivi e Janio Matsuura (administradores do CEC)
  \end{itemize}
\end{frame}

\end{document}


% vim:set ts=2 expandtab:
