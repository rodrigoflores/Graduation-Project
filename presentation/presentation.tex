\documentclass{beamer}
\usepackage[brazil]{babel}
%\usepackage[latin1]{inputenc}
\usepackage[utf8x]{inputenc} 
%\usepackage[all]{xy}
\PrerenderUnicode{ç}

\setbeamercovered{transparent=5}

\usetheme{Antibes}

\title{BOINC + R: Executando rotinas de bioinformática em grades oportunistas}

\author{Rodrigo L. M. Flores \\ \url{flores@ime.usp.br}}


\institute{Instituto de Matemática e Estatística\\Universidade de São Paulo}

\begin{document}

\date{\today}

\frame{\titlepage}

\frame{\tableofcontents}

\section{Introdução}

\begin{frame}
  \frametitle{Bioinformática}
  \begin{itemize}
    \item Aplicação de técnicas computacionais e matemáticas para generação, gerenciamento e análise de bioinformação 
    \item Análise de expressão de genes de dados, Clustering
    \item Algoritmos combinatórios e custosos
  \end{itemize} 
\end{frame}

\begin{frame}
  \frametitle{Computação em grade}
  \begin{itemize}
    \item Clusters dedicados: caros e específicos;
    \item Computadores pessoais em um laboratório ociosos na maior parte do tempo
    \item Não há necessidade de se comprar novos equipamentos
  \end{itemize}
\end{frame}

\begin{frame}
  \frametitle{Computação voluntária}
  \begin{itemize}
    \item Slogan: ``Doe seus ciclos de CPU para um projeto'';
    \item Normalmente projetos com apelo ``humanitário'' (pesquisas de enovelamento de proteínas, malária, descoberta de vida extraterrestre, resolução de instâncias grandes do N-Rainhas, etc);
    \item Computação em grade cuja rede em comum é a Internet; 
    \item Vontade de entrar na rede é do dono do computador;
    \item É possível configurar para que os ciclos de CPU sejam doados se o computador estiver ocioso, se estiver funcionando na energia elétrica, ou programar perfis diferentes em horários ou dias da semana
    \item Exemplos de projetos: SETI@HOME, Folding At Home, N Queens At Home, World Community Grid
    \item Projetos consolidados e funcionando a anos
  \end{itemize}
\end{frame}

\begin{frame}
  \frametitle{Uso de computação voluntária para processamento grades de computadores}
  \begin{itemize}
    \item Idéia surgiu na Universidade de Extremadura, Espanha
    \item Computadores são de uma empresa ou instituição, e a decisão de como eles devem ser utilizados, não são apenas mas sim de uma instituição
    \item Utilização de uma rede de laboratórios
  \end{itemize}
\end{frame}


\section{Ferramentas utilizadas}

\section{Resultados}

\section{Discussão}

\section{Conclusão}

\end{document}


% vim:set ts=2 expandtab:
